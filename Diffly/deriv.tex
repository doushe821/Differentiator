\documentclass{article}
\usepackage[utf8]{inputenc}
\usepackage{amssymb}
\usepackage[russian]{babel}
\usepackage{amsmath}
\begin{document}

Я всегда тяготел к экзотическим животным — змеям, черепахам. А жаб любил с самого детства. Ведь у них такой «улыбчивый» разрез рта, замечательные глаза с серебристой или золотистой крапинкой — куда там человеческим! Очень миролюбивы и неторопливы — незаменимые качества для хорошего соседа.\newline

\[2.71^{(x*\ln(\tg(x^{5})))}\]
\documentclass{article}
\usepackage[utf8]{inputenc}
\usepackage{amssymb}
\usepackage[russian]{babel}
\usepackage{amsmath}
\begin{document}

Я всегда тяготел к экзотическим животным — змеям, черепахам. А жаб любил с самого детства. Ведь у них такой «улыбчивый» разрез рта, замечательные глаза с серебристой или золотистой крапинкой — куда там человеческим! Очень миролюбивы и неторопливы — незаменимые качества для хорошего соседа.\newline

\[2.71^{(x*\ln(x))}\]

... потому что так работают квантовые рояли:
\[2.71^{(x*\ln(x))}\]

Ну ты просто формулы распиши:
\[(x*\ln(x))\]

Кто не рискует, тот не пьëт шампанского. Ну, шампанское мы не пьëм, но рисковать придëтся:
\[x\]

Ну как есть:
\[\ln(x)\]

- Поручик Ржевский! Вы подлец! Я требую удовлетворения!!!
- Ну, пойдем, противный!
\[x\]
\[((\ln(x)+(\frac{1}{x}*x))*(2.71^{(x*\ln(x))}*\ln(2.71)))\]
\newpage
Here comes the Taylor:\newline
\[2.71^{(x*\ln(x))}\]
\(a_{0}\): \(14.9055\)\newline
\(a_{1}\): \(inf\)\newline
\(a_{2}\): \(8\)\newline
\(a_{3}\): \(16\)\newline
\(a_{4}\): \(32\)\newline
\(a_{5}\): \(-nan\)\newline
\(a_{6}\): \(-nan\)\newline
\(a_{7}\): \(-nan\)\newline
Full MacLoren series:\newline
\[2.71^{(x*\ln(x))}=+14.9055*x^{0}+inf*x^{1}+4*x^{2}+2.66667*x^{3}+1.33333*x^{4}-nan*x^{5}-nan*x^{6}-nan*x^{7}+o(x^{7})\]\newline
\end{document}
В 20-30 городах есть очень хорошие школы, хорошие учителя, но этого очень мало. Допустим, в параллели 10 тысяч школьников, да меньше, 5 тысяч школьников, а попадают в суперклассы, очень много остаются за бортом, потому что, например, в 172 школе – самой лучшей школе Москвы – конкурс 15 человек на место. И в конечном счете, когда проходят все собеседования за бортом остается 50 школьников, которые могли бы учиться в этом классе, но они туда не попадают, потому что имеющийся преподавательский состав в количестве 10 человек на 25 школьников – это уже маловато.
\[2.71^{(x*\ln(\tg(x^{5})))}\]

Вы можете быть скептиками, но я предлагаю согласиться с тем, что это очевидно:
\[(x*\ln(\tg(x^{5})))\]

Предположим, что я не помню, как выглядит критерий. Кстати, предположение верно:
\[x\]

Ежу понятно:
\[\ln(\tg(x^{5}))\]

Старый кок:
\[\tg(x^{5})\]

- Поручик Ржевский! Вы подлец! Я требую удовлетворения!!!
- Ну, пойдем, противный!
\[x^{5}\]

Доказательство: чуем Коши:
\[x\]
\[((\ln(\tg(x^{5}))+(\frac{((5*x^{4})*\frac{1}{\cos(x^{5})^{2}})}{\tg(x^{5})}*x))*(2.71^{(x*\ln(\tg(x^{5})))}*\ln(2.71)))\]
\end{document}